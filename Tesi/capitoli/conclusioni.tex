\chapter{Conclusioni}\label{cap:conclusioni}

\intro{In questo capitolo verranno riportate le conclusioni del lavoro svolto, analizzando i risultati ottenuti e le conoscenze acquisite, 
analizzando i possibili sviluppi futuri degli amvbiti applicativi toccati dal progetto e dando una valutazione personale del lavoro svolto.}

\section{Obiettivi raggiunti e consuntivo finale}\label{sec:conclusioni-obiettivi-consuntivo}

In merito al raggiungimento degli obiettivi prefissati per il tirocinio (in sezione~\ref{sec:obiettivi-requisiti}),
si può confermare la soddisfazione di ciascuno. In particolare prima con lo studio e poi con la realizzazione dell'intero progetto,
i concetti di \glsfirstoccur{\gls{blockchaing}} e di \glsfirstoccur{\gls{smartcontractg}} sono stati acquisiti e compresi, dimostrando con la pratica la comprensione del
L'identità sovrana di \glsfirstoccur{\gls{ssig}} è stato applicato ed implementato in un contesto reale, con la possibilità di
interagire con un'applicazione web e con un'applicazione mobile, dimostrando la sua applicabilità in diversi ambiti.
Questo ha permesso, tramite lo studio e l'implementazione autonoma descritta di standard di identità e di firma digitale, di 
acquisire conoscenze e competenze in ambiti di sicurezza informatica e di crittografia, che sono stati approfonditi e studiati
in modo mirato nel progetto realizzato, garantendo la copertura dei requisiti obbligatori. \\

Più nello specifico, sono stati raggiunti correttamente i seguenti obiettivi obbligatori presenti da Piano di Lavoro di tirocinio:
\begin{itemize}
    \item \underline{\textit{O01}}: descrivere i concetti di base della \textit{blockchain} tra cui la sua architettura, i nodi della rete, la crittografia e il consenso distribuito;
    \item \underline{\textit{O02}}: analizzare il concetto di \textit{smart contract} e il linguaggio \glsfirstoccur{\gls{solidityg}}, con particolare attenzione alle vulnerabilità principali e alle tecniche per evitare errori di programmazione;
    \item \underline{\textit{O03}}: approfondire il funzionamento della firma asimmetrica delle transazioni su catena e la validazione dei blocchi, studiando le tipologie di consenso e le catene più conosciute;
    \item \underline{\textit{O04}}: studiare le tecniche di crittografia utilizzate per garantire la sicurezza e la \textit{privacy} delle informazioni personali nell'ambito della \textit{Self Sovereign Identity} e dei protocolli per la gestione delle identità digitali;
    \item \underline{\textit{O05}}: individuare casi d'uso reali per la \textit{Self Sovereign Identity}, analizzando i consorzi internazionali coinvolti in ambito di ricerca e i finanziamenti europei;
    \item \underline{\textit{O06}}: discutere le sfide e i problemi legati alla \textit{Self Sovereign Identity}, studiando un possibile scenario futuro di applicabilità e basato su \glsfirstoccur{\gls{zkpg}}.
\end{itemize}

Come richiesto dalla sezione degli obiettivi desiderabili presenti nella sezione citata, l'applicazione implementa la libreria 
\glsfirstoccur{\gls{web3jsg}} per l'interazione con la \textit{blockchain}, permettendo di interagire con lo \textit{smart contract} in modo corretto.
Inoltre, l'applicazione è stata completamente testata, come riportato dal capitolo~\ref{cap:verifica-validazione}. 
La stessa creazione dell'applicazione permette di esplorare uno scenario di applicazione dell'ambito della \textit{blockchain} e dell'identità connessa
senza trasmettere informazioni personali, risultando così in un'applicazione che rispetta la \textit{privacy} degli utenti.

\clearpage 

Più nello specifico, sono stati raggiunti i seguenti obiettivi desiderabili presenti nel Piano di Lavoro:
\begin{itemize}
    \item \underline{\textit{D01}}: implementare una \glsfirstoccur{\gls{dappg}} (o DApp) tramite le librerie \glsfirstoccur{\gls{ethersjsg}} oppure \glsfirstoccur{\gls{web3jsg}}, utilizzando uno \textit{smart contract} di esempio;
    \item \underline{\textit{D02}}: realizzare una UI (interfaccia utente) per la DApp utilizzando \textit{HTML}, \textit{CSS} e \textit{JavaScript};
    \item \underline{\textit{D03}}: utilizzare la \textit{Self Sovereign Identity} e i protocolli studiati per implementare funzionalità di autenticazione utente nella \textit{DApp};
    \item \underline{\textit{D04}}: testare l'applicazione implementata;
    \item \underline{\textit{D05}}: discutere problemi e sfide relativi all'implementazione di applicazioni decentralizzate su \textit{blockchain} \glsfirstoccur{\gls{ethereumg}}, con particolare attenzione alla scalabilità e alla sicurezza.
\end{itemize}

\section{Conoscenze acquisite e analisi del lavoro svolto}\label{sec:conclusioni-conoscenze-lavoro}

Il tirocinio svolto ha soddisfatto delle aspettative principalmente da un punto di vista conoscitivo, dato che ho capito le implicazioni del mondo \textit{blockchain} e come questo rappresenti,
se ben applicato, un passo importante da un punto di vista di sicurezza. 
Il punto più rilevante di questo tirocinio è stata certamente l'implementazione di questo progetto sfruttando un codice di un laureando magistrale, ampliando questa applicazione a degli standard \glsfirstoccur{\gls{w3cg}}
di non poca importanza, come \glsfirstoccur{\gls{didg}} e \glsfirstoccur{\gls{vcg}}, sviluppando una parte non completamente normata come \glsfirstoccur{\gls{zkpg}}. \\

L'attività è stata molto impegnativa e, nonostante il supporto di massima presente, la parte analitica e di effettivo sviluppo del progetto, specie della parte effettivamente più difficile come 
\textit{Zero Knowledge Proof} sono risultate estremamente teoriche e senza un vero supporto, se non da un punto di vista di puro ragionamento logico, da parte del laureando magistrale Alessio De Biasi, 
il progetto è risultato di difficile implementazione e comprensione autonoma, sia da un punto di vista di requisiti che di codifica vera e propria. Le scelte implementative e la loro realizzazione sono risultate attività pesanti e tediose, 
ma sono state comunque affrontate con successo e con determinazione, permettendomi di acquisire conoscenze e competenze in ambiti di sicurezza informatica e di crittografia decisamente importanti per un percorso di Laurea Triennale come il mio. \\

Di massima, è possibile descrivere le principali conoscenze maturate nell'ambito progettuale:
\begin{itemize}
    \item \textbf{comprensione dell'ambito \textit{blockchain} e sviluppo di \textit{smart contract}}: l'interazione di uno \textit{smart contract} e la sua interazione con un'applicazione web e mobile,
    permettono di comprendere come implementare una struttura dati pubblica ed immutabile, cambiando l'idea stessa di programmazione classica e facendo ben comprendere
    come ogni azione fatta in codice possa avere potenziali implicazioni di sicurezza. Ogni transazione è infatti pubblica e richiede una vera comprensione della struttura sottostante 
    e di come scrivere il codice in modo sicuro e corretto, per evitare perdite di dati o di denaro. In questo ambito, utile certamente l'apprendimento autonomo del linguaggio \textit{Solidity} e della libreria \textit{web3.js},
    che permettono di interagire con la \textit{blockchain} in modo semplice e adattando tale logica a quella di un'applicazione realmente utilizzabile;
    \item \textbf{analisi di standard di identità digitale e di protocolli di firma digitale}: l'implementazione di \textit{Decentralized Identifiers (DID)} e \textit{Verifiable Credentials (VC)} ha permesso di comprendere come sia possibile implementare un sistema di identità digitale
    partendo da implementazioni definite e preesistenti, tuttora sporadicamente applicate, ma con un ambito di applicazione molto ampio. Di fatto, in un mondo come quello odierno dove la \textit{privacy} rappresenta una giusta preoccupazione per gli utenti medi e per la stessa informatica,
    tale progetto dimostra come, in modo relativamente semplice, sia possibile certificare che l'utente sia chi dice di essere, senza trasmettere informazioni personali, ma solo una firma digitale, che può essere verificata da chiunque sul meccanismo di catena di fiducia
    creato e descritto. Il meccanismo di riconoscimento dell'autenticità delle informazioni trasmesse dall'utente ha richiesto una creazione personalizzata di quanto esistente in ambito sicurezza e \textit{blockchain}, creando un meccanismo conforme agli standard 
    e di facile utilizzo, data la complessità degli argomenti descritti e da me trattati, che richiedono una conoscenza approfondita di basi crittografiche, di sicurezza e di come implementare questi standard in modo corretto e concreto;
    \item \textbf{studio e analisi di \textit{Zero Knowledge Proof}}: la realizzazione e l'applicazione di \textit{Zero Knowledge Proof} è stata una delle parti più complesse da implementare, data la natura estremamente teorica della stessa e della 
    sua successiva applicazione, che richiede necessariamente, dati gli standard precedentemente descritti, la comprensione di standard di firma digitale spesso non accuratamente documentati e che richiedono di studiare a fondo gli standard di riferimento e una ricerca
    in gran parte teorica, ben maggiore di quella prevista da alcuni ambiti di studio, dato il tempo relativamente limitato di effettiva implementazione conforme agli standard.
    \item \textbf{nuove competenze di programmazione e progettuali}: l'analisi e la progettazione dei dettaglio di un'applicazione web e mobile, con l'implementazione di un \glsfirstoccur{\gls{backendg}} e di un \glsfirstoccur{\gls{frontendg}}hanno permesso di sviluppare ulteriormente le mie conoscenze apprese durante il corso di Ingegneria
    del Software, normando in modo più efficace le attività di codifica e di realizzazione della documentazione, risolvendo sul campo problemi progettuali ma anche concettuali presenti e sviluppando modifiche in modo agile, separando le responsabilità delle componenti e delle pagine, 
    migliorando così la manutenibilità e la comprensione del codice. 
    L'applicazione del linguaggio di programmazione \textit{TypeScript} unito alla comprensione di vulnerabilità legate al linguaggio \textit{Solidity} risulta utile nell'ulteriore comprensione e applicazione da un punto di vista di sviluppo di applicativi web e mobile, 
    analizzando in dettaglio le caratteristiche offerte da questi linguaggi e librerie e strutturando il codice attraverso \textit{design pattern} di riferimento e architetture utili da un punto di vista logico e di sviluppo attraverso la \textit{Continuous Integration} e la \textit{Continuous Delivery} (\cite{site:cicd}), così implementando modifiche in modo continuativo 
    e strutturato, a calendario e nel corso dell'intero progetto. 
\end{itemize}

\section{Scenari di applicabilità e sviluppi futuri}\label{sec:conclusioni-conoscenze-sviluppi}

L'applicazione sviluppata rappresenta un importante punto di partenza per lo sviluppo di un sistema di identità digitale decentralizzato, che può essere utilizzato in molti contesti reali, come ad esempio:
\begin{itemize}
    \item \textbf{pagamenti anonimi e sicuri}: l'implementazione di \textit{Zero Knowledge Proof} consente di rendere anonime le transazioni effettuate sulla \textit{blockchain}, permettendo di creare un sistema di pagamento decentralizzato, non divulgando informazioni personali ma anche 
    permettendo, data la sua natura pubblica ed immutabile, di riconoscere l'autenticità del pagamento effettuato, invalidando in ogni momento la possibilità di effettuare un doppio pagamento, che può essere verificato da chiunque sul meccanismo di catena di fiducia creato e descritto.
    In questo senso, molte banche cercano di interessarsi al mondo \textit{blockchain} per creare dei sistemi di pagamento \textit{KYC~- Know Your Customer}, che permette di sapere con certezza se l'utente è chi dice di essere, senza trasmettere informazioni personali, ma solo una firma digitale
    pubblicamente verificabile, prevenendo possibili problemi legati al riciclaggio di denaro in modo conforme agli standard;
    \item \textbf{protezione dell'identità ed autenticazione sicura}: il progetto realizzato permette all'utente di costruirsi un'identità digitale `sovrana', permettendo facilmente di compiere un autenticazione senza rivelare alcun dato personale, non dipendendo da piattaforme terze 
    e permettendo di autenticarsi in modo sicuro e conforme agli standard, verificato in modo dettagliato e privo di manipolazioni, partendo da un meccanismo di base come quello previsto;
    \item \textbf{protezione dei dati di archiviazione}: l'utilizzo di \textit{Zero Knowledge Proof} consente di rendere anonime le informazioni archiviate sulla \textit{blockchain}, permettendo di creare un sistema di archiviazione decentralizzato, proibendo a qualsiasi utente 
    non autorizzato di vedere i dati e potervi accedere, permettendo l'accesso alle informazioni solo chi è in grado di dimostrare la conoscenza dello `schema' di accesso comune tramite la presentazione di una credenziale sicura generata sul momento che attesta l'autenticità dell'utente
    e viene verificata secondo sistemi di firma digitale come quelli in oggetto. Nel caso in cui l'utente cerchi di manipolare questi dati, i meccanismi di firma digitale implementati sono in grado di riconoscere qualora 
    l'utente abbia tentato di modificare i dati, invalidando la firma e rendendo impossibile l'accesso ai dati.
\end{itemize}

In questo senso, esistono vari progetti europei che mirano a studiare e scoprire nuove tecnologie per la creazione di sistemi di identità digitale basati su \textit{blockchain}, partendo dal programma comune \textit{Horizon 2020} (\cite{site:horizon2020}),
che ha stanziato un \textit{budget} di 80 miliardi di euro per la ricerca e lo sviluppo delle tecnologie qui in oggetto nel periodo dal 2014 al 2020.
Possiamo citare ad esempio \textit{DECODE~- Decentralized Citizen Owned Data Ecosystem} (\cite{site:decode}), che mira a creare un'infrastruttura di dati decentralizzata e sicura per i cittadini europei
utilizzando tecnologie \textit{blockchain} e \textit{Self Sovereign Identity}, permettendo ad ogni utente di firmare digitalmente le informazioni di identità e condividerle solo con le parti autorizzate.
Un esempio ulteriore è \textit{IRMA~- I Reveal My Attributes} (\cite{site:irma}), sistema di identità digitale basato su \textit{blockchain} che utilizza una combinazione di tecnologie crittografiche mirate a garantire la sicurezza e la \textit{privacy} dei dati personali attraverso l'uso di un'applicazione mobile 
che funge da portafoglio digitale per l'identità dell'utente, generando una prova crittografica verificabile senza dover rivelare alcuna informazione personale. \\

In questo contesto, il meccanismo descritto si inserisce in modo naturale in progetti di questo tipo, inserendosi in parallelo e rappresentando su piccola scala un possibile punto di partenza per lo sviluppo di un sistema di identità digitale decentralizzato vero e proprio.
L'utilizzo parallelo di \textit{Zero Knowledge Proof} e \textit{Self Sovereign Identity} realizza un meccanismo di autenticazione e riconoscimento garantendo il completo controllo delle informazioni trasmesse e senza divulgare dati personali.
Data infatti una prova di autenticazione, l'utente prova la propria identità attraverso la presentazioni di credenziali pubbliche e firmate digitalmente; il meccanismo \textit{blockchain} consente di verificare la firma e quindi l'autenticità dell'utente
mentre i meccanismi di firma sottostanti permettono di riconoscere se l'utente è in grado di dimostrare la conoscenza di un segreto, senza rivelarlo, e quindi di accedere ai dati. \\

Lo scoglio più grande è dato dalla difficoltà di utilizzo di queste tecnologie, che richiedono un'infrastruttura di base che permetta di comprendere a fondo le implicazioni e i vantaggi che portano, ad oggi ancora largamente inesplorati e poco conosciuti, se non da un punto di vista teorico e di ricerca.
Le premesse sulla carta sono ottime: garantire agli utenti il livello di accesso minimo necessario, esponendo l'utente solo a ciò che è strettamente necessario, segmentando l'accesso e controllando l'autenticità delle sue operazioni.
In questo contesto, è fondamentale introdurre un sistema di crittografia sicuro, che garantisca l'implementazione di un sistema a conoscenza zero, in cui è possibile sapere con certezza che ci si può fidare di ciascun nodo presente sulla rete senza conoscere 
alcun dato dei singoli nodi, ma verificando tramite meccanismi riconosciuti e progressivamente standardizzati la loro identità. 
In un mondo in cui la protezione dei dati è diventata una preoccupazione rilevante anche per l'utente medio, occorre necessariamente cercare di stare al passo sfruttando dei meccanismi sicuri che non dipendono da terzi; la \textit{blockchain} riesce a garantire questo livello di sicurezza,
permettendo di creare un sistema di identità digitale decentralizzato e sicuro, in cui l'utente è l'unico detentore delle proprie informazioni e può decidere in autonomia a chi e quando concedere l'accesso. \\

A livello infrastrutturale, è necessario creare dei sistemi che siano in grado di sostenere un sistema di questo tipo, che richiede un'infrastruttura di base basata su \textit{blockchain}, di per sé dispendiosa da un punto di vista energetico e computazionale in modo consistente rispetto a sistemi più tradizionali.
Inoltre, andrebbe certamente standardizzata in maniera rilevante: ad oggi esistono alcune norme ISO che descrivono i termini e le definizioni di base, ma non esistono ancora standard che definiscano in modo chiaro e preciso i meccanismi di autenticazione e di firma digitale,
in larga parte molto teorici in ambito crittografico e non ancora applicati in modo pratico. Legalmente infatti si sta cercando di equiparare questi sistemi a quelli tradizionali, ma data la stessa natura decentralizzata di questo mondo, è difficile trovare un punto di riferimento
univoco ed eticamente accettabile per tutti, come per il mondo dell'intelligenza artificiale. Come per quest'ultimo, rimane principalmente uno strumento in grado potenzialmente di garantire una sicurezza ben maggiore di quella di altri sistemi esistenti ed in grado 
di garantire il vero anonimato nel mondo della rete e di venire incontro egualmente agli utenti finali e alle aziende. 

\section{Valutazione personale}\label{sec:conclusioni-valutazione}

L'attività di stage si è rivelata molto utile per la crescita personale e professionale, permettendo di acquisire nuove conoscenze
e competenze in ambiti di sicurezza informatica e di crittografia, che sono stati approfonditi e studiati in modo mirato nel progetto realizzato.
Inoltre, l'ambito \textit{blockchain} è stato studiato in modo approfondito, permettendo di esplorare in autonomia un campo con ripercussioni
decisamente interessanti nel mondo dell'informatica, decisamente non visti né praticati in corsi universitari ed è stata un'esperienza assolutamente rilevante. 
Ho potuto in questo ambito confrontarmi su aspetti interessanti che offrono nuove prospettive in ambito di sicurezza e che danno una visione delle tecnologie informatiche più consapevole. \\

Spesso ambiti sconosciuti possono essere rilevanti da un punto di vista professionale, e questo è stato un esempio di come
andare oltre i preconcetti dati da un ambito ampiamente teorico e non ormato, si riveli in realtà un'occasione di crescita
e di apprendimento, richiedendo necessariamente di praticare a fondo uno studio mirato delle attività svolte. 
In questi casi, la passione per il conoscere guida e anticipa la necessità di apprendere, dando un'importante possibilità di conoscere e maturare,
altrimenti non offerta in modo così vicino alle realtà aziendali e future rimanendo in ambito puramente accademico.
L'attività di supporto è comunque stata presente di massima, dando ampia possibilità di sviluppo e organizzazione in modo autonomo, 
rimanendo vicino a me come studente e alle mie necessità. \\

Il progetto per me rappresenta una crescita che matura quanto compreso e visto in questi ultimi anni, permettendo di applicare in modo nuovo 
concetti per la maggior parte sconosciuti e con tecnologie nuove, calandomi in un contesto attuale e futuro, usando delle librerie specifiche e sviluppando un
pensiero critico nell'analisi, affinando ulteriormente una visione d'insieme dei corsi fin qui frequentati nel mio percorso di Laurea Triennale, perfezionando il mio metodo di studio 
ed il mio modo di affrontare un progetto di codifica con ripercussioni attuali e future in modo così intenso, completando quanto realizzato nel corso di Ingegneria del Software
e usando tali metodologie per affrontare e suddividere il lavoro in modo autonomo, partendo da quanto appreso in questi mesi.
Quanto studiato rappresenta un importante partenza per il mio futuro percorso di Laurea Magistrale, adattando già ora un codice oggetto di tesi da parte di un laureando magistrale,
ampliando delle competenze trasversali certamente utili per il mio futuro, accademico e non, ed aggiornandomi su un ambito decisamente attuale e di sicuro interesse per possibili sviluppi futuri.