\chapter{Descrizione dello stage}\label{cap:descrizione-stage}

\intro{In questo capitolo viene descritto il contesto in 
cui si è svolto lo stage, il progetto di stage e 
gli obiettivi prefissati in base alla pianificazione iniziale.}

\section{Analisi preventiva dei rischi}

Durante la fase di analisi iniziale sono stati individuati alcuni possibili rischi a cui si potrà andare incontro.
Si è quindi proceduto a elaborare delle possibili soluzioni per far fronte a tali rischi.\\

\textcolor{red}{Da scrivere, prendendo esempio da questo sotto}

\begin{risk}{Performance del simulatore hardware}
    \riskdescription{le performance del simulatore hardware e la comunicazione con questo potrebbero risultare lenti o non abbastanza buoni da causare il fallimento dei test}
    \risksolution{coinvolgimento del responsabile a capo del progetto relativo il simulatore hardware}
\label{risk:hardware-simulator} 
\end{risk}

\section{Obiettivi preposti}

Il tirocinio prevede lo svolgimento dei seguenti obiettivi, riportando questa notazione, come dal documento \textit{Piano di Lavoro}:
\begin{itemize}
    \item O per i requisiti obbligatori, vincolanti in quanto obiettivo primario richiesto dal committente;
    \item D per i requisiti desiderabili, non vincolanti o strettamente necessari, ma dal riconoscibile valore aggiunto;
    \item F per i requisiti facoltativi, rappresentanti valore aggiunto non strettamente competitivo.
\end{itemize}
Le sigle precedentemente indicate saranno seguite da una coppia sequenziale di numeri, identificativo del requisito.

\begin{itemize}

    \item Obbligatori:
        \begin{itemize}
            \item \underline{\textit{O01}}: Descrivere i concetti di base della blockchain, tra cui la sua architettura, i nodi della rete, la criptografia e il consenso distribuito;
            \item \underline{\textit{O02}}: Analizzare il concetto di Smart contract e il linguaggio Solidity, con particolare attenzione alle vulnerabilità principali e alle tecniche per evitare errori di programmazione;
            \item \underline{\textit{O03}}: Approfondire il funzionamento della firma asimmetrica delle transazioni su catena e la validazione dei blocchi, studiando le tipologie di consenso e le catene più conosciute;
            \item \underline{\textit{O04}}: Studiare le tecniche di crittografia utilizzate per garantire la sicurezza e la privacy delle informazioni personali nell'ambito della Self-Sovereign Identity (SSI) e dei protocolli per la gestione delle identità digitali;
            \item \underline{\textit{O05}}: Individuare casi d'uso reali per la SSI, analizzando i consorzi internazionali coinvolti in ambito di ricerca e i finanziamenti europei;
            \item \underline{\textit{O06}}: Discutere le sfide e i problemi legati alla SSI, studiando un possibile scenario futuro di applicabilità e basato su Zero Knowledge Proof (ZKP).
        \end{itemize}

    \item Desiderabili:
        \begin{itemize}
            \item \underline{\textit{D01}}: Implementare una dApp tramite le librerie EthersJS/Web3JS, utilizzando un smart contract di esempio;
            \item \underline{\textit{D02}}: Realizzare una UI (interfaccia utente) per la dApp utilizzando HTML, CSS e JavaScript;
            \item \underline{\textit{D03}}: Utilizzare la SSI e i protocolli studiati per implementare funzionalità di autenticazione utente nella dApp;
            \item \underline{\textit{D04}}: Testare e debuggare la dApp implementata;
            \item \underline{\textit{D05}}: Discutere problemi e sfide relativi all'implementazione di applicazioni decentralizzate su blockchain Ethereum, con particolare attenzione alla scalabilità e alla sicurezza.
        \end{itemize}

    \item Facoltativi:
        \begin{itemize}
            \item \underline{\textit{F01}}: Approfondire l'utilizzo di altri linguaggi di programmazione per gli smart contract, come Vyper;
            \item \underline{\textit{F02}}: Analizzare l'utilizzo di altre tecnologie blockchain, come Polkadot o Cardano, per implementare la SSI;
            \item \underline{\textit{F03}}: Esplorare altre funzionalità delle librerie EthersJS/Web3JS, come l'invio di transazioni;
            \item \underline{\textit{F04}}: Investigare i limiti e le sfide legate all'utilizzo della tecnologia blockchain.
        \end{itemize}
    
\end{itemize}


\section{Pianificazione}
\textcolor{red}{Da scrivere}
