\chapter{Descrizione dello stage}\label{cap:descrizione-stage}

\intro{In questa sezione viene presentata un'analisi del percorso di stage ad alto livello,
individuando possibili rischii e problematiche che potrebbero presentarsi durante lo svolgimento dello stesso.
Inoltre, viene fornita una panoramica degli obiettivi da raggiungere e della pianificazione delle ore di lavoro.}

\section{Analisi preventiva dei rischi}

Durante la fase di analisi iniziale sono stati individuati alcuni possibili rischi 
a cui si potrà andare incontro. Si è quindi proceduto a elaborare delle possibili soluzioni per far fronte a tali rischi.

\begin{risk}{Inesperienza tecnologica e metodologica} 
    \riskdescription{il progetto prevede l'utilizzo di tecnologie e metodologie di cui non si ha piena esperienza e conoscenza, 
    rendendo più difficoltosa la comprensione e l'applicazione delle stesse in fase di implementazione}
    \risksolution{ è stato previsto un periodo di formazione iniziale per studiare le tecnologie e le metodologie da utilizzare
    e l'aiuto/supporto del tutor aziendale e di altri stagisti nella risoluzione di problemi e nella discussione di possibili soluzioni}
\label{risk:inesperienza-tecnologica} 
\end{risk}

\begin{risk}{Difficoltà di integrazione con lo smart contract da usare come libreria}
    \riskdescription{il sistema potrebbe incontrare difficoltà nell'integrazione con lo smart contract da richiamare per la gestione delle identità digitali,
    a causa di errori di programmazione o di problemi di comunicazione tra le parti coinvolte}
    \risksolution{è stato previsto un periodo di test e debug per verificare il corretto funzionamento del sistema e per risolvere eventuali problemi riscontrati,
    approfondendo il dialogo con il tutor aziendale e con lo stagista che ha sviluppato lo smart contract che dovrà essere richiamato in fase di implementazione}
    \label{risk:integrazione-smart-contract}
\end{risk}

\begin{risk}{Ritardi nello sviluppo}
    \riskdescription{potrebbero verificarsi ritardi nello sviluppo del sistema a causa di problemi tecnici o imprevisti, come la mancanza di risorse necessarie o la complessità del progetto,
    rimodulando adeguatamente attività e tempistiche}
    \label{risk:ritardi-sviluppo}
\end{risk}

\begin{risk}{Problemi di sicurezza}
    \riskdescription{il sistema potrebbe riscontrare problemi di sicurezza e vulnerabilità a attacchi informatici o la mancanza di protezione dei dati sensibili}
    \risksolution{fin dall'inizio dell'attività di sviluppo, si cerca di garantire un'implementazione al passo con le \textit{best practices} previste dai linguaggi di programmazione utilizzate e lato web, 
    come l'utilizzo di librerie e framework aggiornati e sicuri, la validazione dei dati in input e la protezione da attacchi di tipo \textit{SQL injection} e \textit{XSS}. Inoltre, è possibile prevedere un'attività di testing approfondito,
    ad esempio utilizzando strumenti di analisi statica del codice e di penetration testing, per verificare la corretta implementazione delle misure di sicurezza e la presenza di eventuali vulnerabilità}
    \label{risk:problemi-sicurezza}
\end{risk}

\begin{risk}{Cambiamenti dei requisiti durante lo sviluppo}
    \riskdescription{i requisiti del sistema potrebbero cambiare durante l'attività di implementazione, ad esempio a causa di una modifica delle esigenze del progetto o di un errore di analisi iniziale}
    \risksolution{è possibile prevedere un'attività di pianificazione flessibile, ad esempio utilizzando metodologie agili, come \glsfirstoccur{\gls{scrumg}}, che prevedono una pianificazione adattiva ai cambiamenti dei requisiti. 
    Inoltre, potrebbe essere utile prevedere una comunicazione costante con il tutor aziendale e gli altri stagisti in fase di validazione dei requisiti aggiornati, procedendo con lo sviluppo in modo coerente ed organizzato}
    \label{risk:cambiamenti-requisiti}
\end{risk}

\section{Obiettivi e requisiti}

Il tirocinio prevede lo svolgimento dei seguenti obiettivi, riportando questa notazione, come dal documento \textit{Piano di Lavoro}:
\begin{itemize}
    \item O per i requisiti obbligatori, vincolanti in quanto obiettivo primario richiesto dal committente;
    \item D per i requisiti desiderabili, non vincolanti o strettamente necessari, ma dal riconoscibile valore aggiunto;
    \item F per i requisiti facoltativi, rappresentanti valore aggiunto non strettamente competitivo.
\end{itemize}
Le sigle precedentemente indicate saranno seguite da una coppia sequenziale di numeri, identificativo del requisito.

\begin{itemize}

    \item Obbligatori:
        \begin{itemize}
            \item \underline{\textit{O01}}: Descrivere i concetti di base della blockchain, tra cui la sua architettura, i nodi della rete, la criptografia e il consenso distribuito;
            \item \underline{\textit{O02}}: Analizzare il concetto di Smart contract e il linguaggio Solidity, con particolare attenzione alle vulnerabilità principali e alle tecniche per evitare errori di programmazione;
            \item \underline{\textit{O03}}: Approfondire il funzionamento della firma asimmetrica delle transazioni su catena e la validazione dei blocchi, studiando le tipologie di consenso e le catene più conosciute;
            \item \underline{\textit{O04}}: Studiare le tecniche di crittografia utilizzate per garantire la sicurezza e la privacy delle informazioni personali nell'ambito della Self-Sovereign Identity (SSI) e dei protocolli per la gestione delle identità digitali;
            \item \underline{\textit{O05}}: Individuare casi d'uso reali per la SSI, analizzando i consorzi internazionali coinvolti in ambito di ricerca e i finanziamenti europei;
            \item \underline{\textit{O06}}: Discutere le sfide e i problemi legati alla SSI, studiando un possibile scenario futuro di applicabilità e basato su Zero Knowledge Proof (ZKP).
        \end{itemize}

    \item Desiderabili:
        \begin{itemize}
            \item \underline{\textit{D01}}: Implementare una dApp tramite le librerie EthersJS/Web3JS, utilizzando un smart contract di esempio;
            \item \underline{\textit{D02}}: Realizzare una UI (interfaccia utente) per la dApp utilizzando HTML, CSS e JavaScript;
            \item \underline{\textit{D03}}: Utilizzare la SSI e i protocolli studiati per implementare funzionalità di autenticazione utente nella dApp;
            \item \underline{\textit{D04}}: Testare e debuggare la dApp implementata;
            \item \underline{\textit{D05}}: Discutere problemi e sfide relativi all'implementazione di applicazioni decentralizzate su blockchain Ethereum, con particolare attenzione alla scalabilità e alla sicurezza.
        \end{itemize}

    \item Facoltativi:
        \begin{itemize}
            \item \underline{\textit{F01}}: Approfondire l'utilizzo di altri linguaggi di programmazione per gli smart contract, come Vyper;
            \item \underline{\textit{F02}}: Analizzare l'utilizzo di altre tecnologie blockchain, come Polkadot o Cardano, per implementare la SSI;
            \item \underline{\textit{F03}}: Esplorare altre funzionalità delle librerie EthersJS/Web3JS, come l'invio di transazioni;
            \item \underline{\textit{F04}}: Investigare i limiti e le sfide legate all'utilizzo della tecnologia blockchain.
        \end{itemize}
    
\end{itemize}

\section{Pianificazione}
Il tirocinio ha una durata di 11 settimane per la durata complessiva di 320 ore, con una suddivisione del tempo suddivisa 
in part-time orizzontale per le prime 6 settimane (dal 20 Marzo al 30 Aprile) e full-time per le ultime 5 settimane (dal 2 Maggio al 7 Giugno).\\
La ripartizione temporale delle attività è la seguente:
\begin{itemize}
    \item{Prima Settimana: \textbf{Introduzione alle tecnologie blockchain} (20 ore)}
    \begin{itemize}
        \item Studio dei concetti di base della blockchain, tra cui la sua architettura, i nodi della rete, la crittografia e il consenso distribuito;
    \end{itemize}
    \item{Seconda Settimana: \textbf{Concetti avanzati di blockchain} (20 ore)}
    \begin{itemize}
        \item Studio del concetto di Smart contract e del linguaggio Solidity, con particolare attenzione alle vulnerabilità principali e alle tecniche per evitare errori di programmazione;
    \end{itemize}
    \item{Terza Settimana: \textbf{Token, scalabilità e immutabilità} (20 ore)}
    \begin{itemize}
        \item Studio del funzionamento della firma asimmetrica delle transazioni su catena e della validazione dei blocchi, studiando le tipologie di consenso e le catene più conosciute;
    \end{itemize}
    \item{Quarta Settimana: \textbf{Introduzione alla Self-Sovereign Identity (SSI)} (20 ore)}
    \begin{itemize}
        \item Analisi delle tecniche di crittografia utilizzate per garantire la sicurezza e la privacy delle informazioni personali nell'ambito della Self-Sovereign Identity (SSI) e dei protocolli per la gestione delle identità digitali;
    \end{itemize}
    \item{Quinta Settimana: \textbf{Introduzione a Zero Knowledge Proof (ZKP)} (20 ore)}
    \begin{itemize}
        \item Studio delle tecniche di \glsfirstoccur{\gls{zkpg}} e delle loro applicazioni nell'ambito della \glsfirstoccur{\gls{ssig}};
    \end{itemize}
    \item{Sesta Settimana: \textbf{Studio delle librerie ethers.js-web3.js} (20 ore)}
    \begin{itemize}
        \item Implementazione esempi pratici con librerie \glsfirstoccur{\gls{ethersjsg}} e \glsfirstoccur{\gls{web3jsg}} per l'interazione con la blockchain Ethereum;
    \end{itemize}
    \item{Settima Settimana: \textbf{Implementazione Proof of Concept} (40 ore)}
    \begin{itemize}
        \item Analisi ed implementazione interfaccia utente e configuraizione base interazione con blockchain e smart contract;
    \end{itemize}
    \item{Ottava Settimana: \textbf{Implementazione soluzioni SSI e ZKP nel Proof of Concept} (40 ore)}
    \begin{itemize}
        \item Implementazione funzionalità di autenticazione utente e di gestione delle identità digitali;
    \end{itemize}
    \item{Nona Settimana: \textbf{Studio soluzioni second layer e prima analisi sicurezza} (40 ore)}
    \begin{itemize}
        \item Analisi soluzioni di scalabilità second layer e implementazione di una soluzione di autenticazione basata su ZKP;
    \end{itemize}
    \item{Decima Settimana: \textbf{Analisi di vulnerabilità e implementazione meccanismi di sicurezza} (40 ore)}
    \begin{itemize}
        \item Analisi vulnerabilità e implementazione meccanismi di sicurezza per la gestione delle identità digitali;
    \end{itemize}
    \item{Undicesima Settimana: \textbf{Finanziamenti/Consorzi coinvolti e discussione scenari di applicabilità} (40 ore)}
    \begin{itemize}
        \item Analisi dei finanziamenti e dei consorzi coinvolti nello sviluppo di soluzioni \glsfirstoccur{\gls{ssig}} e discussione scenari di applicabilità;
    \end{itemize}
\end{itemize}