% Glossary entries
\newglossaryentry{apig} {
    name=\glslink{apig}{Application Program Interface},
    text=Application Program Interface,
    sort=api,
    description={In informatica con il termine \emph{Application Programming Interface API} 
    (ing. interfaccia di programmazione di un'applicazione) si indica ogni insieme di procedure disponibili al programmatore, 
    di solito raggruppate a formare un set di strumenti specifici per l'espletamento di un determinato compito all'interno 
    di un certo programma. La finalità è ottenere un'astrazione, di solito tra l'hardware e il programmatore 
    o tra software a basso e quello ad alto livello semplificando così il lavoro di programmazione}
}

\newglossaryentry{agileg}{
    name=\glslink{agileg}{Agile}
    text=Agile,
    sort=agile,
    description={In ingegneria del software, 
    con il termine \emph{Agile} si indica un insieme di metodi di sviluppo del software emersi 
    a partire dai primi anni 2000 e fondati su un insieme di principi comuni,
    direttamente o indirettamente derivati dai principi del \emph{Manifesto per lo sviluppo agile del software} 
    (ing. \emph{Manifesto for Agile Software Development}). 
    Il termine \emph{agile} si riferisce a un insieme di metodi di sviluppo del software basati 
    su processi iterativi ed incrementali, dove i requisiti e le soluzioni evolvono 
    attraverso la collaborazione tra individui auto-organizzati e interfunzionanti}
}

\newglossaryentry{scrumg}{
    name=\glslink{scrumg}{Scrum},
    text=Scrum,
    sort=scrum,
    description={In ingegneria del software, \emph{Scrum} è una metodologia di sviluppo 
    iterativa ed incrementale per la gestione del ciclo di sviluppo del software, 
    iterativa in quanto il lavoro viene suddiviso in blocchi (sprint) e
    incrementale perché il lavoro viene suddiviso in parti che vengono consegnate in modo incrementale. 
    Il termine \emph{scrum} deriva dal rugby, dove indica una formazione composta dalla linea
    dei giocatori che si fronteggiano e si accaparrano il pallone con le gambe, cercando di spingerlo verso la meta avversaria}
}

\newglossaryentry{gitg}{
    name=\glslink{gitg}{Git},
    text=Git,
    sort=git,
    description={\emph{Git} è un software di controllo versione distribuito utilizzabile da interfaccia a riga di comando, 
    creato da Linus Torvalds nel 2005. 
    Lo scopo di Git è quello di gestire progetti con velocità e semplicità, 
    garantendo allo stesso tempo la possibilità di gestire flussi di lavoro complessi 
    sulla base di un sistema di controllo di versione non lineare e distribuito. 
    Git permette di tenere traccia di tutte le modifiche apportate al codice sorgente 
    di un progetto sviluppato da più persone e di coordinarle}
}

\newglossaryentry{softwareg}{
    name=\glslink{softwareg}{Software},
    text=software,
    sort=software,
    description={In informatica con il termine \emph{software} si intende l'insieme delle componenti immateriali 
    di un sistema elettronico di elaborazione dati, comprese le istruzioni (programmi) 
    e i dati di supporto (documentazione e dati di configurazione). 
    Il termine è stato coniato da Alan Turing e compare per la prima volta nel suo articolo 
    del 1935 \emph{On Computable Numbers, with an Application to the Entscheidungsproblem}}
}

\newglossaryentry{umlg} {
    name=\glslink{umlg}{Unified Modeling Language},
    text=UML,
    sort=uml,
    description={In ingegneria del software \emph{UML, Unified Modeling Language} (ing. linguaggio di modellazione unificato) è un linguaggio di modellazione e specifica basato sul paradigma object-oriented. 
    L'\emph{UML} svolge un'importantissima funzione di ``lingua franca'' nella comunità della progettazione e programmazione a oggetti. 
    Gran parte della letteratura di settore usa tale linguaggio per descrivere soluzioni analitiche e progettuali in modo sintetico e comprensibile a un vasto pubblico}
}

\newglossaryentry{ideg}{
    name=\glslink{ideg}{Integrated Development Environment},
    text=IDE,
    sort=ide,
    description={In informatica con il termine \emph{Integrated Development Environment} (ing. ambiente di sviluppo integrato) si indica un software che, in fase di programmazione, supporta i programmatori nello sviluppo del codice sorgente di un programma. 
    Solitamente un \emph{IDE} è composto da un editor di codice sorgente, un compilatore ed un debugger. 
    Inoltre, spesso, fornisce strumenti per l'automazione di alcune operazioni ripetitive, 
    per la navigazione all'interno del codice e per semplificare alcune operazioni di sviluppo}
}

\newglossaryentry{blockchaing}{
    name=\glslink{blockchaing}{Blockchain},
    text=Blockchain,
    sort=blockchain,
    description={In informatica con il termine \emph{Blockchain} si indica una struttura dati condivisa e immutabile.
    La blockchain è resa immutabile dall'utilizzo di funzioni crittografiche di hash e dalla struttura dati a blocchi concatenati.
    ed è condivisa in quanto è distribuita in una rete peer-to-peer. }
}

\newglossaryentry{pocg}{
    name=\glslink{pocg}{Proof of Concept}
    text=Proof of Concept,
    sort=proof of concept,
    description={In ingegneria del software con il termine \emph{Proof of Concept} (ing. prova di fattibilità) si indica un'implementazione o esperimento che dimostra la fattibilità di un concetto o di un'idea. 
    Il \emph{Proof of Concept} è generalmente focalizzato su singoli aspetti o caratteristiche del progetto, 
    dimostrando che il concetto o l'idea è praticabile e funzionante}
}

\newglossaryentry{w3cg}{
    name=\glslink{w3cg}{World Wide Web Consortium},
    text=W3C,
    sort=w3c,
    description={Il \emph{World Wide Web Consortium} (W3C) è un'organizzazione internazionale che ha come scopo quello di sviluppare tutte le potenzialità del World Wide Web. 
    Il W3C produce e promuove standard tecnici aperti e liberi per il Web, 
    allo scopo di garantirne la crescita a lungo termine}
}

\newglossaryentry{ssig}{
    name=\glslink{ssig}{Self Sovereign Identity},
    text=Self Sovereign Identity,
    sort=ssi,
    description={In informatica con il termine \emph{Self-Sovereign Identity} (ing. identità autonoma) si indica un modello di identità digitale che permette ad un individuo di avere il controllo completo delle proprie informazioni personali, 
    senza doverle condividere con terze parti. 
    L'identità autonoma è basata su tecnologie decentralizzate, come la blockchain, 
    e permette di creare un'identità digitale che non può essere controllata da nessuno, 
    nemmeno da un'autorità centrale}
}

\newglossaryentry{didg}{
    name=\glslink{didg}{Decentralized Identifier},
    text=Decentralized Identifier,
    sort=decentralized identifier,
    description={In informatica con il termine \emph{Decentralized Identifier} (ing. identificatore decentralizzato) si indica un identificatore univoco, 
    che può essere utilizzato per identificare un'entità digitale, come una persona, un'organizzazione o un dispositivo. 
    Un \emph{DID} è un URI che fa riferimento ad un documento che contiene le informazioni relative all'entità digitale identificata}
}

\newglossaryentry{vcg}{
    name=\glslink{vcg}{Verifiable Credentials},
    text=Verifiable Credentials,
    sort=vc,
    description={In informatica con il termine \emph{Verifiable Credential} (ing. credenziale verificabile) si indica un documento digitale che contiene informazioni relative ad un'entità digitale, 
    come una persona, un'organizzazione o un dispositivo. 
    Una \emph{VC} è un documento firmato digitalmente da un'autorità che ne certifica l'autenticità e che può essere verificato da terze parti}
}

\newglossaryentry{zkpg}{
    name=\glslink{zkpg}{Zero Knowledge Proof},
    text=Zero Knowledge Proof,
    sort=zkp,
    description={In crittografia con il termine \emph{Zero-Knowledge Proof} (ing. prova a conoscenza zero) si indica un protocollo che permette ad un soggetto di dimostrare di conoscere un certo dato, 
    senza doverlo rivelare. 
    Il protocollo permette di dimostrare che un certo dato è vero, senza dover rivelare il dato stesso. 
    In questo modo è possibile dimostrare di conoscere un dato, senza doverlo rivelare}
}