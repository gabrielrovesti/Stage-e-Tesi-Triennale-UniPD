% Acronyms
\newacronym[description={\glslink{apig}{Application Program Interface}}]
    {api}{API}{Application Program Interface}

\newacronym[description={\glslink{umlg}{Unified Modeling Language}}]
    {uml}{UML}{Unified Modeling Language}

\newacronym[description={\glslink{scrumg}{Scrum}}]
    {scrum}{Scrum}{Scrum}

\newacronym[description={\glslink{agile}{Agile}}]
    {agile}{Agile}{Agile}

\newacronym[description={\glslink{software}{Software}}]
    {software}{Software}{Software}

\newacronym[description={\glslink{gitg}{Git}}]
    {git}{Git}{Git}

\newacronym[description={\glslink{ideg}{Integrated Development Environment}}]
    {ide}{IDE}{Integrated Development Environment}


% Glossary entries
\newglossaryentry{apig} {
    name=\glslink{api}{API},
    text=Application Program Interface,
    sort=api,
    description={in informatica con il termine \emph{Application Programming Interface API} 
    (ing. interfaccia di programmazione di un'applicazione) si indica ogni insieme di procedure disponibili al programmatore, 
    di solito raggruppate a formare un set di strumenti specifici per l'espletamento di un determinato compito all'interno 
    di un certo programma. La finalità è ottenere un'astrazione, di solito tra l'hardware e il programmatore 
    o tra software a basso e quello ad alto livello semplificando così il lavoro di programmazione}
}

\newglossaryentry{agileg}{
    name=\glslink{agile}{Agile},
    text=Agile,
    sort=agile,
    description={in ingegneria del software, 
    con il termine \emph{Agile} si indica un insieme di metodi di sviluppo del software emersi 
    a partire dai primi anni 2000 e fondati su un insieme di principi comuni,
    direttamente o indirettamente derivati dai principi del \emph{Manifesto per lo sviluppo agile del software} 
    (ing. \emph{Manifesto for Agile Software Development}). 
    Il termine \emph{agile} si riferisce a un insieme di metodi di sviluppo del software basati 
    su processi iterativi ed incrementali, dove i requisiti e le soluzioni evolvono 
    attraverso la collaborazione tra individui auto-organizzati e interfunzionanti}
}

\newglossaryentry{scrumg}{
    name=\glslink{scrum}{Scrum},
    text=Scrum,
    sort=scrum,
    description={in ingegneria del software, \emph{Scrum} è una metodologia di sviluppo 
    iterativa ed incrementale per la gestione del ciclo di sviluppo del software, 
    iterativa in quanto il lavoro viene suddiviso in blocchi (sprint) e
    incrementale perché il lavoro viene suddiviso in parti che vengono consegnate in modo incrementale. 
    Il termine \emph{scrum} deriva dal rugby, dove indica una formazione composta dalla linea
    dei giocatori che si fronteggiano e si accaparrano il pallone con le gambe, cercando di spingerlo verso la meta avversaria}
}

\newglossaryentry{gitg}{
    name=\glslink{git}{Git},
    text=Git,
    sort=git,
    description={\emph{Git} è un software di controllo versione distribuito utilizzabile da interfaccia a riga di comando, 
    creato da Linus Torvalds nel 2005. 
    Lo scopo di Git è quello di gestire progetti con velocità e semplicità, 
    garantendo allo stesso tempo la possibilità di gestire flussi di lavoro complessi 
    sulla base di un sistema di controllo di versione non lineare e distribuito. 
    Git permette di tenere traccia di tutte le modifiche apportate al codice sorgente 
    di un progetto sviluppato da più persone e di coordinarle}
}

\newglossaryentry{softwareg}{
    name=\glslink{software}{Software},
    text=software,
    sort=software,
    description={in informatica con il termine \emph{software} si intende l'insieme delle componenti immateriali 
    di un sistema elettronico di elaborazione dati, comprese le istruzioni (programmi) 
    e i dati di supporto (documentazione e dati di configurazione). 
    Il termine è stato coniato da Alan Turing e compare per la prima volta nel suo articolo 
    del 1935 \emph{On Computable Numbers, with an Application to the Entscheidungsproblem}}
}

\newglossaryentry{umlg} {
    name=\glslink{uml}{UML},
    text=UML,
    sort=uml,
    description={in ingegneria del software \emph{UML, Unified Modeling Language} (ing. linguaggio di modellazione unificato) è un linguaggio di modellazione e specifica basato sul paradigma object-oriented. 
    L'\emph{UML} svolge un'importantissima funzione di ``lingua franca'' nella comunità della progettazione e programmazione a oggetti. 
    Gran parte della letteratura di settore usa tale linguaggio per descrivere soluzioni analitiche e progettuali in modo sintetico e comprensibile a un vasto pubblico}
}

\newglossaryentry{ideg}{
    name=\glslink{ide}{IDE},
    text=IDE,
    sort=ide,
    description={in informatica con il termine \emph{Integrated Development Environment} (ing. ambiente di sviluppo integrato) si indica un software che, in fase di programmazione, supporta i programmatori nello sviluppo del codice sorgente di un programma. 
    Solitamente un \emph{IDE} è composto da un editor di codice sorgente, un compilatore ed un debugger. 
    Inoltre, spesso, fornisce strumenti per l'automazione di alcune operazioni ripetitive, 
    per la navigazione all'interno del codice e per semplificare alcune operazioni di sviluppo}
}
