%----------------------------------------------------------------------------------------
%	DESCRIPTION OF THE INTERACTION BETWEEN THE STUDENT AND THE INTERNAL TUTOR
%----------------------------------------------------------------------------------------
\section*{Tutor aziendale ed interazione con lo studente}
% Personalizzare definendo le modalità di interazione col tutor aziendale

Durante lo stage, lo studente avrà come tutor aziendale \nomeTutorAziendale\ \cognomeTutorAziendale, che lo guiderà e lo supporterà nell'approfondimento delle tecnologie oggetto di studio. 
Il tutor aziendale sarà il responsabile della supervisione delle attività svolte dallo studente e della valutazione dei risultati ottenuti. 
Questi si prefigge di presentare allo studente l'organizzazione dell'azienda, 
del tirocinio, coinvolgendolo nei progetti presenti e fornendo una panoramica di massima delle tecnologie utilizzate, oggetto di studio e della tesi realizzata.

\medskip

A questo proposito, di comune accordo, lo studente e il tutor aziendale intendono stabilire degli incontri regolari svolti direttamente, 
al fine di valutare congiuntamente i progressi del lavoro svolto, chiarire eventuali dubbi e fornire un feedback sullo stato di avanzamento.
Questi incontri saranno svolti regolarmente, almeno una volta la settimana, e potranno essere svolti anche in modalità telematica con il tutor interno, in base alla necessità ed alla disponibilità di entrambi.
In questo modo, lo studente può essere guidato e supportato nella realizzazione della tesi di laurea, al fine di poterla presentare in modo completo e soddisfacente.

