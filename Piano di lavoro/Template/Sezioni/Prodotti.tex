%----------------------------------------------------------------------------------------
%	DESCRIPTION OF THE PRODUCTS THAT ARE BEING EXPECTED FROM THE STAGE
%----------------------------------------------------------------------------------------
\section*{Prodotti attesi}
% Personalizzare definendo i prodotti attesi (facoltativo)
Durante il tirocinio, lo studente dovrà produrre un Proof of Concept di un'applicazione decentralizzata (dApp) che interagisce con la blockchain Ethereum. La dApp dovrà essere implementata utilizzando la libreria EthersJS/Web3JS per la lettura e la scrittura di dati sulla blockchain. In particolare, la dApp dovrà implementare le seguenti funzionalità:

\begin{itemize}
    \item \textbf{Autenticazione utente}: la dApp dovrà utilizzare i protocolli di identità digitale studiati durante il tirocinio (ad esempio, DID e Verifiable Credentials) per implementare un sistema di autenticazione utente.
    \item \textbf{Registrazione di dati sulla blockchain}: la dApp dovrà permettere agli utenti di registrare dati sulla blockchain Ethereum, utilizzando un'interfaccia utente intuitiva e facile da usare.
    \item \textbf{Lettura di dati dalla blockchain}: la dApp dovrà permettere agli utenti di leggere i dati registrati sulla blockchain Ethereum in modo sicuro e affidabile.
\end{itemize}

\newpage

Inoltre, lo studente dovrà produrre un documento che descriva il percorso di studio realizzato durante il tirocinio. 
In particolare, il documento dovrà includere i seguenti elementi di massima:

\begin{enumerate}
    \item \textbf{Introduzione}: una breve descrizione del progetto e degli obiettivi dello stage.
    \item \textbf{Descrizione della dApp}: una descrizione dettagliata dell'applicazione decentralizzata sviluppata durante il tirocinio, comprese le funzionalità implementate e le tecnologie utilizzate.
    \item \textbf{Analisi delle tecnologie}: una descrizione delle tecnologie utilizzate per implementare la dApp, compresi i protocolli di identità digitale e i protocolli di Zero Knowledge Proof e di Self-Sovereign Identity.
    \item \textbf{Problematiche e soluzioni}: una discussione delle problematiche incontrate durante lo sviluppo della dApp e delle soluzioni adottate per risolverle.
    \item \textbf{Sfide e limiti}: un'analisi critica delle sfide incontrate durante lo sviluppo della dApp e nello studio autonomo delle tecnologie blockchain, comprese le ricerche di sicurezza effettuate e le possibili soluzioni future.
    \item \textbf{Scenari futuri}: una discussione delle possibili implementazioni future delle tecnologie blockchain e delle applicazioni decentralizzate.
    \item \textbf{Conclusioni}: una sintesi delle attività svolte durante il tirocinio e delle competenze acquisite.
\end{enumerate}

Qualora, al termine dell'analisi, lo studente disponga ancora di tempo a sua disposizione, potrà dedicarsi a approfondimenti o implementazioni aggiuntive, concordate con l'azienda ospitante.