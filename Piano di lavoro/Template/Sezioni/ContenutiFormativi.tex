\section*{Contenuti formativi previsti}
% Personalizzare indicando le tecnologie e gli ambiti di interesse dello stage
Durante questo progetto di stage lo studente avrà occasione di approfondire le sue conoscenze in ambito blockchain e self-sovereign identity,
come indicato di seguito in dettaglio.
\begin{itemize}
    \item \textbf{Concetti di base blockchain}
    \begin{itemize}
        \item Studio del funzionamento della blockchain;
        \item Concetto di wallet e funzionamento firma asimmetrica delle transazioni su catena;
        \item Validazione e mining dei blocchi;
        \newpage
        \item Tipologie di Consenso e studio delle catene più conosciute;
        \item Concetto di Smart contract e linguaggio Solidity;
        \item Scalabilità e limiti della tecnologia blockchain;
        \item Tokenizzazione e creazione di token su blockchain;
        \item Concetto di immutabilità nella blockchain e il ruolo della crittografia;
        \item Vulnerabilità principali in Solidity (e.g. reentrancy, integer overflow/underflow, etc.).
    \end{itemize}
    \item \textbf{Self-Sovereign Identity e Zero Knowledge Proof}
    \begin{itemize}
        \item Studio del concetto di self-sovereign identity (SSI);
        \item Descrizione di protocolli e tecnologie per la gestione delle identità digitali;
        \item Analisi delle caratteristiche, delle potenzialità e delle criticità di ciascuna di esse;
        \item Approfondimento delle tecniche di crittografia utilizzate per garantire la sicurezza e la privacy delle informazioni personali nell'ambito della SSI;
        \item Casi d'uso reali individuati;
        \item Consorzi internazionali coinvolti e finanziamenti europei;
        \item Studio di un possibile scenario futuro di applicabilità, discutendo problemi e sfide correlati;
        \item Zero Knowledge Proof: cos'è e come potrebbe servire per la SSI.
    \end{itemize}
    \item \textbf{Studio delle librerie EthersJS/Web3JS e implementazione tramite Proof of Concept (POC)} 
    \begin{itemize}
        \item Studio delle principali librerie Javascript per interagire con la blockchain Ethereum (EthersJS, Web3JS, ecc.);
        \item Analisi delle caratteristiche e delle funzionalità offerte dalle librerie;
        \item Implementazione di un Proof of Concept (POC) con librerie EthersJS/Web3JS;
        \item Configurazione dell'ambiente di sviluppo per interagire con una blockchain Ethereum di test;
        \item Studio dell'architettura di un'applicazione decentralizzata (dApp) su blockchain Ethereum e dei componenti principali (smart contract, frontend, backend);
        \item Implementazione di un'interfaccia utente (UI) per la dApp utilizzando HTML, CSS e Javascript;
        \item Interazione con la blockchain Ethereum tramite la libreria EthersJS/Web3JS per la lettura e la scrittura di dati sulla blockchain;
        \item Implementazione di funzionalità di autenticazione utente utilizzando la SSI e i protocolli studiati in precedenza;
        \item Testing e debugging del POC implementato;
        \item Discussione di problemi e sfide relativi all'implementazione di applicazioni decentralizzate su blockchain Ethereum.
    \end{itemize}
\end{itemize}