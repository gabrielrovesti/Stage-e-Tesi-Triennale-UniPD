%----------------------------------------------------------------------------------------
%   USEFUL COMMANDS
%----------------------------------------------------------------------------------------

\newcommand{\dipartimento}{Dipartimento di Matematica ``Tullio Levi-Civita''}

%----------------------------------------------------------------------------------------
% 	USER DATA
%----------------------------------------------------------------------------------------

% Data di approvazione del piano da parte del tutor interno; nel formato GG Mese AAAA
% compilare inserendo al posto di GG 2 cifre per il giorno, e al posto di 
% AAAA 4 cifre per l'anno
\newcommand{\dataApprovazione}{2023-10-03}

% Dati dello Studente
\newcommand{\nomeStudente}{Gabriel}
\newcommand{\cognomeStudente}{Rovesti}
\newcommand{\matricolaStudente}{2009088}
\newcommand{\emailStudente}{gabriel.rovesti@studenti.unipd.it}
\newcommand{\telStudente}{+ 39 346 68 89 789}

% Dati del Tutor Aziendale
\newcommand{\nomeTutorAziendale}{Fabio}
\newcommand{\cognomeTutorAziendale}{Pallaro}
\newcommand{\emailTutorAziendale}{f.pallaro@synclab.it}
\newcommand{\telTutorAziendale}{+ 39 333 13 68 8500}
\newcommand{\ruoloTutorAziendale}{}

% Dati dell'Azienda
\newcommand{\ragioneSocAzienda}{Sync Lab S.r.l}
\newcommand{\indirizzoAzienda}{Galleria Spagna, 28, Padova (PD)}
\newcommand{\sitoAzienda}{https://www.synclab.it/}
\newcommand{\emailAzienda}{info@synclab.it}
\newcommand{\partitaIVAAzienda}{P.IVA 07952560634}

% Dati del Tutor Interno (Docente)
\newcommand{\titoloTutorInterno}{Prof.ssa}
\newcommand{\nomeTutorInterno}{Ombretta}
\newcommand{\cognomeTutorInterno}{Gaggi}

\newcommand{\prospettoSettimanale}{
     % Personalizzare indicando in lista, i vari task settimana per settimana
     % sostituire a XX il totale ore della settimana
     \begin{enumerate}
        \item \textbf{Prima Settimana - Introduzione alle tecnologie blockchain (20 ore)}
        \begin{itemize}
            \item Introduzione alle tecnologie blockchain;
            \item Studio del funzionamento della blockchain e dei concetti di base (wallet, firma asimmetrica, transazioni, etc.);
            \item Studio dei principali tipi di blockchain e dei meccanismi di consenso;
        \end{itemize}

        \item \textbf{Seconda Settimana - Concetti avanzati di blockchain (20 ore)}
        \begin{itemize}
            \item Studio di concetti avanzati di blockchain come Smart contract, Solidity e tokenizzazione;
            \item Analisi di vulnerabilità principali in Solidity (e.g. reentrancy, integer overflow/underflow, etc.);
            \item Studio della scalabilità e dei limiti della tecnologia blockchain;
        \end{itemize}
        
        \item \textbf{Terza Settimana - Token, scalabilità e immutabilità(20 ore)}
        \begin{itemize}
            \item Studio della tokenizzazione e della creazione di token su blockchain;
            \item Studio della scalabilità e dei limiti della tecnologia blockchain;
            \item Concetto di immutabilità nella blockchain e il ruolo della crittografia;
        \end{itemize}
        
        \item \textbf{Quarta Settimana - Self-sovereign identity (SSI): protocolli, tecnologie e criticità (20 ore)}
        \begin{itemize}
            \item Studio del concetto di self-sovereign identity;
            \item Descrizione di protocolli e tecnologie per la gestione delle identità digitali;
            \item Analisi delle caratteristiche, delle potenzialità e delle criticità di ciascuna di esse;
        \end{itemize}
        
        \newpage 

        \item \textbf{Quinta Settimana - Zero Knowledge Proof (ZKP) e sua applicazione alla SSI (20 ore)}
        \begin{itemize}
            \item Studio del concetto di Zero Knowledge Proof;
            \item Applicazione di Zero Knowledge Proof alla SSI;
            \item Casi d'uso reali della SSI e della tecnologia Zero Knowledge Proof;
            \item Primo studio consorzi internazionali coinvolti e finanziamenti europei;
        \end{itemize}
        
        \item \textbf{Sesta Settimana - Studio delle librerie EthersJS/Web3JS (20 ore)}
        \begin{itemize}
            \item Studio delle principali librerie Javascript per interagire con la blockchain Ethereum (EthersJS, Web3JS, ecc.);
            \item Analisi delle caratteristiche e delle funzionalità offerte dalle librerie;
            \item Configurazione dell'ambiente di sviluppo per interagire con una blockchain Ethereum di test;
        \end{itemize}
        
        \item \textbf{Settima e Ottava Settimana - Implementazione Proof of Concept con EthersJS/Web3JS (40 ore)}
        \begin{itemize}
            \item Definizione dei requisiti per il PoC da implementare;
            \item Studio delle librerie EthersJS/Web3JS per interagire con la blockchain Ethereum;
            \item Configurazione dell'ambiente di sviluppo per interagire con una blockchain Ethereum di test;
            \item Studio dell'architettura di un'applicazione decentralizzata (dApp) su blockchain Ethereum e dei componenti principali (smart contract, frontend, backend);
            \item Implementazione di un'interfaccia utente (UI) per la dApp utilizzando HTML, CSS e Javascript;
            \item Interazione con la blockchain Ethereum tramite la libreria EthersJS/Web3JS per la lettura e la scrittura di dati sulla blockchain;
            \item Implementazione di funzionalità di autenticazione utente utilizzando la SSI e i protocolli studiati in precedenza;
            \item Testing e debugging del POC implementato;
            \item Discussione di problemi e sfide relativi all'implementazione di applicazioni decentralizzate su blockchain Ethereum.
        \end{itemize}

        \newpage

        \item \textbf{Nona Settimana - SSI: approfondimenti e implementazione nel PoC}
        \begin{itemize}
            \item Studio approfondito del concetto di self-sovereign identity (SSI);
            \item Descrizione di protocolli e tecnologie per la gestione delle identità digitali;
            \item Analisi delle caratteristiche, delle potenzialità e delle criticità di ciascuna di esse;
            \item Approfondimento delle tecniche di crittografia utilizzate per garantire la sicurezza e la privacy delle informazioni personali nell'ambito della SSI;
            \item Casi d'uso reali individuati;
            \item Consorzi internazionali coinvolti e finanziamenti europei;
            \item Studio di un possibile scenario futuro di applicabilità, discutendo problemi e sfide correlati;
            \item Integrazione della SSI nel POC sviluppato in precedenza;
            \item Testing e debugging delle funzionalità di autenticazione utente utilizzando la SSI;
            \item Discussione di problemi e sfide relativi all'utilizzo della SSI in applicazioni decentralizzate su blockchain Ethereum.
        \end{itemize}

        \item \textbf{Decima Settimana - ZKP: approfondimenti e implementazione nel PoC}
        \begin{itemize}
            \item Studio di Zero Knowledge Proof (ZKP);
            \item Approfondimento dell'utilizzo di ZKP per la SSI;
            \item Implementazione di una funzionalità di ZKP nel POC sviluppato in precedenza;
            \item Testing e debugging delle funzionalità di ZKP implementate;
            \item Discussione di problemi e sfide relativi all'utilizzo di ZKP in applicazioni decentralizzate su blockchain Ethereum e nella SSI.
        \end{itemize}   

        \item \textbf{Undicesima Settimana - Scalabilità nella blockchain e analisi second layer (20 ore)}
        \begin{itemize}
            \item Studio della scalabilità della tecnologia blockchain;
            \item Analisi dei limiti della blockchain Ethereum e delle possibili soluzioni;
            \item Studio delle soluzioni di second layer per la scalabilità, come gli State Channels e i Sidechains;
            \item Implementazione di una soluzione di second layer nel POC sviluppato in precedenza;
            \item Testing e debugging della soluzione di second layer implementata;
            \newpage 
            \item Implicazioni e limitazioni dell'uso delle soluzioni di second layer in ambito blockchain.
        \end{itemize}

        \item \textbf{Dodicesima Settimana - Analisi di vulnerabilità e sicurezza (20 ore)}
        \begin{itemize}
            \item Studio delle vulnerabilità principali della blockchain Ethereum;
            \item Analisi dei rischi di sicurezza per i contratti intelligenti su Ethereum;
            \item Studio delle tecniche di hacking utilizzate per attaccare le blockchain;
            \item Utilizzo di strumenti di sicurezza per analizzare e proteggere le reti blockchain;
            \item Analisi delle vulnerabilità degli exchange di criptovalute e delle strategie per mitigare i rischi;
            \item Valutazione della sicurezza delle criptovalute e dei portafogli digitali;
        \end{itemize}
        
        \item \textbf{Tredicesima Settimana - Implementazione misure individuate nel PoC (20 ore)}
        \begin{itemize}
            \item Revisione e analisi dei risultati dell'analisi di vulnerabilità e sicurezza della blockchain Ethereum e dei contratti intelligenti;
            \item Implementazione di contromisure per mitigare i rischi di sicurezza identificati durante l'analisi di vulnerabilità;
            \item Studio delle best practices per la sicurezza delle applicazioni decentralizzate (dApp) su blockchain Ethereum;
            \item Valutazione dell'efficacia delle contromisure implementate per mitigare i rischi di sicurezza;
            \item Discussione di nuove minacce di sicurezza per la blockchain Ethereum e dei metodi per affrontarle.
        \end{itemize}
        
        \item \textbf{Quattordicesima Settimana - Analisi di limiti e sfide nell'implementazione decentralizzata (20 ore)}
        \begin{itemize}
            \item Analisi delle principali implementazioni di blockchain (ad esempio, Bitcoin, Ethereum, Ripple, etc.);
            \item Studio delle differenze tra le implementazioni di blockchain pubbliche e private;
            \item Discussione dei limiti attuali della tecnologia blockchain e delle aree in cui è necessario migliorare;
            \item Analisi delle potenziali applicazioni della blockchain in vari settori (ad esempio, finanza, supply chain, sanità, etc.);
            \item Studio delle opportunità e dei rischi associati alla tokenizzazione di asset tradizionali tramite la blockchain;
            \item Discussione delle implicazioni legali e regolamentari della blockchain e delle criptovalute.
        \end{itemize}

        \item \textbf{Quindicesima Settimana - Consorzi internazionali coinvolti e finanziamenti europei (20 ore)}
        \begin{itemize}
            \item Approfondimento sulle organizzazioni di standardizzazione, come ISO e IEEE, e sul loro ruolo nello sviluppo e nella definizione degli standard blockchain;
            \item Studio delle partnership tra aziende e organizzazioni per lo sviluppo di progetti blockchain, come il consorzio R3;
            \item Analisi dei finanziamenti disponibili per lo sviluppo di progetti blockchain a livello nazionale e internazionale, come il programma Horizon 2020 dell'Unione Europea;
            \item Esplorazione delle implicazioni legali e regolatorie nell'ambito della partecipazione a progetti blockchain finanziati da enti pubblici o privati;
            \item Discussione delle strategie di networking e di comunicazione per individuare e partecipare a progetti di ricerca e sviluppo sulla blockchain;
            \item Approfondimento sui processi di selezione dei progetti finanziati, e su come aumentare le probabilità di successo nella partecipazione a bandi pubblici o privati.
        \end{itemize} 

        \item \textbf{Sedicesima Settimana - Discussione scenari di applicabilità e conclusione (20 ore)}
        \begin{itemize}
            \item Analisi dei casi d'uso più interessanti e promettenti per la blockchain e le criptovalute in ambiti specifici come la finanza, la logistica, la salute, la pubblica amministrazione, etc.;
            \item Discussione dei vantaggi e delle limitazioni dell'utilizzo della blockchain e delle criptovalute rispetto alle tecnologie tradizionali;
            \item Esplorazione di nuovi sviluppi tecnologici e di progetti di ricerca in corso che potrebbero influenzare l'evoluzione della blockchain e delle criptovalute;
            \item Valutazione dei possibili impatti socio-economici e delle opportunità per l'innovazione che potrebbero derivare dall'adozione su larga scala della blockchain e delle criptovalute;
            \item Discussione delle sfide e dei rischi che potrebbero sorgere con l'uso della blockchain e delle criptovalute, come la scalabilità, la privacy, la sicurezza e la regolamentazione.
        \end{itemize} 

    \end{enumerate}

}

\newcommand{\totaleOre}{320}

\newcommand{\obiettiviObbligatori}{
    \item \underline{\textit{O01}}: Descrivere i concetti di base della blockchain, tra cui la sua architettura, i nodi della rete, la criptografia e il consenso distribuito;
    \item \underline{\textit{O02}}: Analizzare il concetto di Smart contract e il linguaggio Solidity, con particolare attenzione alle vulnerabilità principali e alle tecniche per evitare errori di programmazione;
    \item \underline{\textit{O03}}: Approfondire il funzionamento della firma asimmetrica delle transazioni su catena e la validazione dei blocchi, studiando le tipologie di consenso e le catene più conosciute;
    \item \underline{\textit{O04}}: Studiare le tecniche di crittografia utilizzate per garantire la sicurezza e la privacy delle informazioni personali nell'ambito della Self-Sovereign Identity (SSI) e dei protocolli per la gestione delle identità digitali;
    \item \underline{\textit{O05}}: Individuare casi d'uso reali per la SSI, analizzando i consorzi internazionali coinvolti in ambito di ricerca e i finanziamenti europei;
    \item \underline{\textit{O06}}: Discutere le sfide e i problemi legati alla SSI, studiando un possibile scenario futuro di applicabilità e basato su Zero Knowledge Proof (ZKP).
}

\newcommand{\obiettiviDesiderabili}{
	\item \underline{\textit{D01}}: Implementare una dApp tramite le librerie EthersJS/Web3JS, utilizzando un smart contract di esempio;
    \item \underline{\textit{D02}}: Realizzare una UI (interfaccia utente) per la dApp utilizzando HTML, CSS e JavaScript;
    \item \underline{\textit{D03}}: Utilizzare la SSI e i protocolli studiati per implementare funzionalità di autenticazione utente nella dApp;
    \item \underline{\textit{D04}}: Testare e debuggare la dApp implementata;
    \item \underline{\textit{D05}}: Discutere problemi e sfide relativi all'implementazione di applicazioni decentralizzate su blockchain Ethereum, con particolare attenzione alla scalabilità e alla sicurezza.
}

\newcommand{\obiettiviFacoltativi}{
	\item \underline{\textit{F01}}: Approfondire l'utilizzo di altri linguaggi di programmazione per gli smart contract, come Vyper;
    \item \underline{\textit{F02}}: Analizzare l'utilizzo di altre tecnologie blockchain, come Polkadot o Cardano, per implementare la SSI;
    \item \underline{\textit{F03}}: Esplorare altre funzionalità delle librerie EthersJS/Web3JS, come l'invio di transazioni;
    \item \underline{\textit{F04}}: Investigare i limiti e le sfide legate all'utilizzo della tecnologia blockchain.
}